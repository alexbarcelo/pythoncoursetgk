\documentclass[12pt,a4paper]{article}
\usepackage[utf8]{inputenc}
\usepackage[catalan]{babel}
\usepackage{amsmath}
\usepackage{amsfonts}
\usepackage{amssymb}
\usepackage{graphicx}
\usepackage{listings}
\author{Alex Barceló}
\title{Curs de Python -- Telecogresca \\ {\sc Aplicació de frases de benvinguda}}
\begin{document}
\maketitle

\setcounter{section}{-1}
\section{Preparació del \texttt{virtualenv}}

Començarem per preparar un \verb+virtualenv+. Això ho farem per assegurar-nos de no ``contaminar'' el sistema amb els paquets (i les versions) que desitgem insta\lgem{}ar.

En el següents exemples suposaré que la carpeta de l'usuari és el directori de treball i el lloc desitjat on treballar.

Si encara no tenim el mòdul \verb+virtualenv+ el podem aconseguir:
\begin{itemize}
\item En \textbf{Ubuntu} a través del paquet \verb+virtualenv+.
\item En general, podem insta\lgem{}ar el mòdul a tot el sistema mitjançant la comanda \verb+sudo pip install virtualenv+
\end{itemize}

Per a crear la carpeta que serà el nostre \emph{virtual environment} només cal executar:

\begin{verbatim}
~ $ virtualenv tgkenv
~ $ source tgkenv/bin/activate
(tgkenv) ~ $ echo "Look mom, I am in a virtualenv!"
\end{verbatim}

La segona comanda s'encarrega d'assegurar que l'intèrpret de Python ``actiu'' serà el de l'entorn virtual, i no el de sistema.

Ja podem obtenir el Django simplement utilitzant la comanda \verb+pip+ (sense \verb+sudo+! aquesta és una de les avantatges del \verb+virtualenv+).

\begin{verbatim}
(tgkenv) ~ $ pip install django==1.9
\end{verbatim}

\section{Inici del projecte}

[Nota: assegureu-vos de tenir el \verb+virtualenv+ actiu, si ho voleu fer així. Això implica realitzar la comanda \verb+source tgkenv/bin/activate+ o similar prèviament]

En \textbf{Django} podem començar l'estructura d'arxius amb la següent comanda:

\begin{verbatim}
(tgkenv) ~ $ django-admin startproject tgkhandson
(tgkenv) ~ $ cd tgkhandson
(tgkenv) ~/tgkhandson $ ./manage.py startapp benvinguda
\end{verbatim}

Després de realitzar les comandes anteriors haureu obtingut una estructura d'arxius que contindrà un projecte Django, amb una configuració inicial força raonable, i també haureu crear l'estructura d'arxius de l'aplicació de \verb+benvinguda+ que és la que programarem en aquest \emph{hands-on}.
\end{document}