\documentclass[12pt,a4paper]{article}
\usepackage[utf8]{inputenc}
\usepackage[catalan]{babel}
\usepackage{amsmath}
\usepackage{amsfonts}
\usepackage{amssymb}
\usepackage{graphicx}
\usepackage{listings}
\usepackage{url}
\usepackage{color}
\usepackage{fullpage}

\definecolor{mygreen}{rgb}{0,0.6,0}
\definecolor{mygray}{rgb}{0.5,0.5,0.5}
\definecolor{mymauve}{rgb}{0.58,0,0.82}

\lstset{ %
  backgroundcolor=\color{white},   % choose the background color; you must add \usepackage{color} or \usepackage{xcolor}
  basicstyle=\footnotesize\ttfamily, % the size of the fonts that are used for the code
  breakatwhitespace=false,         % sets if automatic breaks should only happen at whitespace
  breaklines=true,                 % sets automatic line breaking
  captionpos=b,                    % sets the caption-position to bottom
  commentstyle=\color{mygreen},    % comment style
  frame=single,	                   % adds a frame around the code
  keepspaces=true,                 % keeps spaces in text, useful for keeping indentation of code (possibly needs columns=flexible)
  keywordstyle=\color{blue},       % keyword style
  language=Python,                 % the language of the code
  numbers=left,                    % where to put the line-numbers; possible values are (none, left, right)
  numbersep=5pt,                   % how far the line-numbers are from the code
  numberstyle=\tiny\color{mygray}, % the style that is used for the line-numbers
  rulecolor=\color{black},         % if not set, the frame-color may be changed on line-breaks within not-black text (e.g. comments (green here))
  showspaces=false,                % show spaces everywhere adding particular underscores; it overrides 'showstringspaces'
  showstringspaces=false,          % underline spaces within strings only
  showtabs=false,                  % show tabs within strings adding particular underscores
  stringstyle=\color{mymauve},     % string literal style
  tabsize=2,	                   % sets default tabsize to 2 spaces
  title=\lstname                   % show the filename of files included with \lstinputlisting; also try caption instead of title
}

\author{Alex Barceló}
\title{Curs de Python -- Telecogresca \\ {\sc Aplicació de frases de benvinguda}}
\begin{document}
\maketitle

\setcounter{section}{-1}
\section{Preparació del \texttt{virtualenv}}

Començarem per preparar un \verb+virtualenv+. Això ho farem per assegurar-nos de no ``contaminar'' el sistema amb els paquets (i les versions) que desitgem insta\lgem{}ar.

En el següents exemples suposaré que la carpeta de l'usuari és el directori de treball i el lloc desitjat on treballar.

Si encara no tenim el mòdul \verb+virtualenv+ el podem aconseguir:
\begin{itemize}
\item En \textbf{Ubuntu} a través del paquet \verb+virtualenv+.
\item En general, podem insta\lgem{}ar el mòdul a tot el sistema mitjançant la comanda \verb+sudo pip install virtualenv+
\end{itemize}

Per a crear la carpeta que serà el nostre \emph{virtual environment} només cal executar:

\begin{verbatim}
~ $ virtualenv tgkenv
~ $ source tgkenv/bin/activate
(tgkenv) ~ $ echo "Look mom, I am in a virtualenv!"
\end{verbatim}

La segona comanda s'encarrega d'assegurar que l'intèrpret de Python ``actiu'' serà el de l'entorn virtual, i no el de sistema.

Ja podem obtenir el Django simplement utilitzant la comanda \verb+pip+ (sense \verb+sudo+! aquesta és una de les avantatges del \verb+virtualenv+).

\begin{verbatim}
(tgkenv) ~ $ pip install django==1.9
\end{verbatim}

\section{Inici del projecte}

[Nota: assegureu-vos de tenir el \verb+virtualenv+ actiu, si ho voleu fer així. Això implica realitzar la comanda \verb+source tgkenv/bin/activate+ o similar prèviament, i fer-ho en totes les consoles que utilitzeu]

En \textbf{Django} podem començar l'estructura d'arxius amb la següent comanda:

\begin{verbatim}
(tgkenv) ~ $ django-admin startproject tgkhandson
(tgkenv) ~ $ cd tgkhandson
(tgkenv) ~/tgkhandson $ ./manage.py startapp benvinguda
\end{verbatim}

Després de realitzar les comandes anteriors haureu obtingut una estructura d'arxius que contindrà un projecte Django, amb una configuració inicial força raonable, i també haureu crear l'estructura d'arxius de l'aplicació de \verb+benvinguda+ que és la que programarem en aquest \emph{hands-on}.

Per a indicar que volem utilitzar l'aplicació \verb+benvinguda+ cal obrir l'arxiu \verb+settings.py+ (carpeta \verb+tgkhandson+) i modificar la variable \verb+INSTALLED_APPS+, concretament:

\begin{lstlisting}
INSTALLED_APPS = [
    'benvinguda',
    'django.contrib.admin',
    ...  # deixar la resta de linies iguals
    
\end{lstlisting}

\section{Definició del model de dades}

El primer que necessitarem és definir el model de dades que Django que s'utilitzarà per a guardar les frases de benvinguda. No cal tenir coneixements de SQL, de bases de dades, o similars; n'hi ha prou utilitzant els \verb+model+ de Django.

El model de dades serà molt senzill, en el nostre cas. Només utilitzarem un cap \verb+missatge+, i deixarem (tot i que avui no mostrarem com utilitzar-ho) un camp \verb+only_anonymous+ que ens permetria en un futur filtrar les frases de benvinguda en funció si l'usuari està loggejat o encara no.

Els models es defineixen a l'arxiu \verb+models.py+ (carpeta \verb+benvinguda+). El contingut del model pot ser el següent:

\begin{lstlisting}
class WelcomeFrase(models.Model):
    """Sentences to greet the user.

    The application contains a list of sentences and, 
    randomly, one will be used in order to greet the 
    user (at the main page).
    """
    class Meta(object):
        verbose_name = "frase de benvinguda"
        verbose_name_plural = "frases de benvinguda"

    missatge = models.CharField('missatge', max_length=100)
    only_anonymous = models.BooleanField('nomes per anonims', 
                                         default=True)
\end{lstlisting}

Un cop hem definit el model (que és una simple classe Python) procedim a indicar a Django que volem utilitzar aquest model de dades. El que internament fa és crear les taules a la base de dades\footnote{En el nostre cas, Django estarà utilitzant una base de dades \textbf{SQLite}, que és molt adecuada per fer proves i per a desenvolupament però mai s'ha d'utilitzar en producció}, amb els seus camps, relacions, etc.

El procediment és senzill:

\begin{verbatim}
(tgkenv) ~/tgkhandson $ ./manage.py makemigrations
(tgkenv) ~/tgkhandson $ ./manage.py migrate
\end{verbatim}

L'operació està dividida en una preparació (\verb+makemigrations+) que comprova la consistència dels models i que la operació sigui correcte i l'execució dels canvis (\verb+migrate+). Per a una aplicació real, el \verb+makemigrations+ el realitzaria el desenvolupador i un cop comprovada la correctesa, l'administrador executaria \verb+migrate+ en el servidor i la base de dades de producció.

\section{Introduïnt frases}

Django té una interfície d'administració molt potent, i activada per defecte en les darreres versions. Per a utilitzar-la hem d'indicar que desitgem que el model de dades introduït prèviament sigui utilitzat en l'admin. Per a fer-ho hem d'afegir (deixeu les línies ja existents) el següent a l'arxiu \verb+benvinguda/admin.py+:

\begin{lstlisting}
from .models import WelcomeFrase

admin.site.register(WelcomeFrase)
\end{lstlisting}

Creem un usuari administrador i iniciem el servidor de proves:

\begin{verbatim}
(tgkenv) ~/tgkhandson $ ./manage.py createsuperuser
Username (leave blank to use 'pacorandom'): admin
Email address: admin@telecogresca.com
Password: gre3skam0la
Password (again): gre3skam0la
Superuser created successfully.
(tgkenv) ~/tgkhandson $ ./manage.py runserver
\end{verbatim}

I ara ja podem accedir a la URL \url{http://127.0.0.1:8000/admin/} (usuari \verb+admin+, contrasenya \verb+gr3skam0la+, o el que hagueu posat) i introduir frases de benvinguda.

\section{Preparant Controller i View}

A continuació prepararem:
\begin{description}
\item[Controller] escull una frase aleatòria (o seguint certs criteris)
\item[View] mostra al visitant la web, que contindrà la frase escollida
\end{description}

Vigileu que l'estructura clàssica Model-View-Controller en Django es conserva, però s'anomenen Model-Template-View. L'abús de la paraula \emph{view} sovint ocasiona certes confusions.

Comencem repassant el codi del \verb+views.py+ (el típicament anomenat \emph{Controller}, carpeta \verb+benvinguda+):

\begin{lstlisting}
from django.views.generic import TemplateView
from benvinguda.models import WelcomeFrase
from random import choice

class HomeView(TemplateView):
    """First page view."""
    template_name = 'home.html'

    def get(self, request, *args, **kwargs):
        """Put the 'welcome' in context, template will show it."""
        context = {
            'welcome': choice(WelcomeFrase.objects.all()).missatge,
        }
        return self.render_to_response(context)
\end{lstlisting}

Veiem una selecció aleatòria molt simple. Un cop definida la variable \verb+context+, la feina és ara del \emph{template} per a generar el HTML que anirà a l'usuari (típicament, la capa de \emph{view} en un Model-View-Controller). L'arxiu s'ha d'anomenar \verb+home.html+ i el crearem a la carpeta \verb+benvinguda/templates+ amb el següent contingut:

\begin{verbatim}
<h1>{{ welcome }} </br><small>esca, esca, esca, Telecogresca!</small></h1>
\end{verbatim}

Sou lliures de millorar aquest template, però aquí utilitzo una mostra minimalista de template HTML quasi-vàlid (els navegadors ho acceptaran, tot i que no és un document vàlid en el seu estat actual).

\section{Rutes}

Ja quasi hem acabat, però si us hi fixeu encara no hem indicat quina URL haurà de mostrar la benvinguda. La relació entre URL i controladors associats s'acostuma a fer mitjançant l'arxiu \emph{urls.py} (carpeta \verb+tgkhandson+). La complexitat de les rutes pot ser molt alta, però per a l'exemple que estem realitzant, hi haurà prou modificant l'arxiu fins arribar al següent estat:

\begin{lstlisting}
from django.conf.urls import url
from django.contrib import admin
from benvinguda.views import HomeView

urlpatterns = [
    url(r'^benvinguda/', HomeView.as_view()),
    url(r'^admin/', admin.site.urls),
]
\end{lstlisting}

Per veure l'aplicació acabada només cal fer (si heu apagat el servidor):
\begin{verbatim}
(tgkenv) ~/tgkhandson $ ./manage.py runserver
\end{verbatim}

I ara ja podem accedir a la URL \url{http://127.0.0.1:8000/benvinguda/} i podrem veure les frases de benvinguda.

\end{document}